\documentclass[11pt]{exam}
\usepackage{amsmath, amsthm}  
\usepackage{enumitem}
\usepackage[T1]{fontenc}
\usepackage[bitstream-charter]{mathdesign}
\usepackage{parskip}
\usepackage{pgfplots}
\usepackage{booktabs}
\usepackage{tabularx}
\pgfplotsset{soldot/.style={color=blue,only marks,mark=*}} \pgfplotsset{holdot/.style={color=blue,fill=white,only marks,mark=*}}
\pgfplotsset{compat=1.17} 
% Theorems
\theoremstyle{definition}
\newtheorem*{defi}{Definition}
\newtheorem*{thm}{Theorem}

% Special sets
\newcommand{\C}{\mathbb{C}}
\newcommand{\CP}{\mathbb{CP}}
\newcommand{\GG}{\mathbb{G}}
\newcommand{\N}{\mathbb{N}}
\newcommand{\Q}{\mathbb{Q}}
\newcommand{\R}{\mathbb{R}}
\newcommand{\RP}{\mathbb{RP}}
\newcommand{\T}{\mathbb{T}}
\newcommand{\Z}{\mathbb{Z}}
\renewcommand{\H}{\mathbb{H}}

\pagenumbering{gobble}
\newcounter{questionCounter}
\newenvironment{numedquestion}[0]{%
	\stepcounter{questionCounter}%
    \vspace{.2in}%
        \ifx\writtensection\undefined
        \noindent{\bf \questiontype \; \arabic{questionCounter}. }%
        \else
          \if\writtensection
          \noindent{\bf \questiontype \; \arabic{questionCounter}. }%
          \else
          \noindent{\bf \questiontype \; \writtensection.\arabic{questionCounter} }%
        \fi
    \vspace{0.3em} \hrule \vspace{.1in}%
}{}



\footer{}{\thepage}{}
\renewcommand{\frac}{\dfrac}

\usepackage{multicol}

\newcommand{\myclass}{Calculus with Exercises A}
\newcommand{\myname}{DO LE DUY}
\newcommand{\myhwtype}{Report 11}
\newcommand{\questiontype}{Problem}
\newcommand{\writtensection}{10}
\newtheorem*{prop}{Proposition}
\usepackage{pgf,tikz,pgfplots}
\pgfplotsset{compat=1.15}
\usepackage{mathrsfs}
\usetikzlibrary{arrows}
\headrule
\header{{\myname}}%
{\emph{\myclass}}%
{\textbf{\myhwtype }}

\begin{document}
\thispagestyle{empty}
\begin{center}
  {\Large \textbf{\myclass{} — \myhwtype{} }} \\
  {\myname{} } \\
  \today
\end{center}

\begin{numedquestion}
    (1) This integral is a definition of the \emph{Gamma Function}. 
    \begin{itemize}
        \item Because we have:
        \begin{align*}
            \Gamma(n+1) &=\int_{0}^{\infty} x^{n} e^{-x} d x \\
            &=\left[-x^{n} e^{-x}\right]_{0}^{\infty}+\int_{0}^{\infty} n x^{n-1} e^{-x} d x \\
            &=\lim _{x \rightarrow \infty}\left(-x^{n} e^{-x}\right)-\left(-0^{n} e^{-0}\right)+n \int_{0}^{\infty} x^{n-1} e^{-x} d x \\
            &=n \int_{0}^{\infty} x^{n-1} e^{-x} d x =n \Gamma(n) \\
            \Gamma(1) &=\int_{0}^{\infty} x^{1-1} e^{-x} d x \\
            &=\left[-e^{-x}\right]_{0}^{\infty} \\
            &=\lim _{x \rightarrow \infty}\left(-e^{-x}\right)-\left(-e^{-0}\right) \\
            &=0-(-1) =1.
        \end{align*}
        We would expect the final result: $\Gamma(n+1) = \int_{0}^{\infty} x^{n} e^{-x} d x = n!$. We would get it using an induction proof argument: 
        \begin{itemize}
            \item \textbf{Base Case}: We already had $\Gamma(1) = 0! = 1$. 
            \item \textbf{Induction Step}: Assume that $\Gamma(n) = \int^{\infty}_0 x^{n-1} e^{-x} d x = (n-1)!$, then \\
            $\Gamma(n+1) = \int^{\infty}_0 x^{n} e^{-x} d x = n!$. This was already demonstrated above.
            \item \textbf{Conclusion}: We could conclude that \[\int_{0}^{\infty} x^{n} e^{-x} d x = n!\] 
        \end{itemize}
    \end{itemize}
    
    (2)
    \begin{align*}
        \int_{0}^{1} \frac{1}{1-x^{4}} d x = \int_0^1 \left(\frac{a}{1-x} + \frac{b}{1+x} + \frac{cx + d}{1+x^{2}}\right) d x.
    \end{align*} 
    We have 
    \[
        \begin{cases}
            x^3: a - b - c = 0 \\
            x^2: a + b - d = 0 \\
            x: a - b + c = 0 \\
            1: a + b + d = 1
        \end{cases}
        \implies  
        \begin{cases}
            a = b = 0.25\\
            c = 0 \\
            d = 0.5.    
        \end{cases}
    \]
    Then, 
    \begin{align*}
        \int_{0}^{1} \frac{1}{1-x^{4}} d x = \int_0^1 \left( \frac{1}{4(1-x)} + \frac{1}{4(1+x)} + \frac{1}{2(1+x^{2})} \right) d x.
    \end{align*} 
    But \[
        \int_0^1 \frac{1}{4(1-x)} d x = \lim_{x_1 \rightarrow 1} \left(\int_0^{x_1} \frac{1}{1-x} d x\right) = \lim_{x_1 \rightarrow 1} (-\ln (1-x_1)),
    \]
    which diverges.
    We could conclude that the integral diverges. 

    
\end{numedquestion} 

\begin{numedquestion}
    \begin{prop}
        Let \(f\) be a positive continuous function on \([1, \infty)\) such that \(\lim _{x \rightarrow \infty} f(x)=\alpha .\) Show that if the integral
        \[
        \int_{1}^{\infty} \frac{f(x)}{x} d x
        \]
        converges, then \(\alpha\) must be zero. 
    \end{prop} 
    We will use contradiction to prove the above proposition. Assume that \(\lim _{x \rightarrow \infty} f(x)=\alpha,\) where $\alpha$ is an arbitrary positive number. Then there must exist an $x_0 \geq 1$ such that for all $x \geq x_0$, $f(x)$ is in the neighborhood $(\alpha - t, \alpha +t)$ for a positive $t: \alpha - t  > 0$. Then, we have:
    \[
        \int_1^\infty \frac{f(x)}{x} \geq \int_{x_0}^{\infty} \frac{f(x)}{x}  \geq \int_{x_0}^{\infty} \frac{\alpha -t}{x} = (\alpha -t)(\ln \infty - \ln x_0)
    \], which does not converge. Thus, we have reached a contradiction. The argument is similar for $\alpha < 0$. We could conclude that $\alpha$ must be zero for $\int_1^\infty \frac{f(x)}{x}$ to converge. 
\end{numedquestion}  

\begin{numedquestion}
    We will find $p$ in two ways. 
    \begin{itemize}
        \item 
        We will use Taylor expansion of $\sin x$ to find $p$. 
        \begin{align*}
            \sin (x) &=0+1 x+0 x^{2}+\frac{-1}{3 !} x^{3}+0 x^{4}+\cdots \\
            &=x-\frac{x^{3}}{3 !}+\frac{x^{5}}{5 !}-\frac{x^{7}}{7 !}+\cdots
        \end{align*}
        Then: 
        \begin{align*}
        &\int_{0}^{1} \frac{\sin x}{x^{p}} d x \\
        =& x^{1-p}-\frac{x^{3-p}}{3 !}+\frac{x^{5-p}}{5 !}-\frac{x^{7-p}}{7 !}+\cdots
        \end{align*}
        The integral is improper for $p>1$. From the p-Series Theorem, we could conclude that for $1<p<2$ the integral is converges, and diverges for $p \geq 2$. 
        \newline
        
        \item It is easy to note that the integral is improper for $p>1$. We will just use simple squeeze theorem to find $p$ that the integral converges or diverges. \\

\definecolor{zzttff}{rgb}{0.6,0.2,1}
\definecolor{ccqqqq}{rgb}{0.8,0,0}
\definecolor{qqwuqq}{rgb}{0,0.39215686274509803,0}
\begin{tikzpicture}[line cap=round,line join=round,>=triangle 45,x=1cm,y=1cm]
\begin{axis}[
x=2cm,y=2cm,
axis lines=middle,
ymajorgrids=true,
xmajorgrids=true,
xmin=-0.7352332360639006,
xmax=4.037704851297755,
ymin=-1.0325900134725907,
ymax=2.1747374028638378,]
\clip(-0.7352332360639006,-1.0325900134725907) rectangle (4.037704851297755,2.1747374028638378);
\draw[line width=2pt,color=qqwuqq,smooth,samples=100,domain=-0.7352332360639006:4.037704851297755] plot(\x,{sin(((\x))*180/pi)});
\draw[line width=2pt,color=ccqqqq,smooth,samples=100,domain=-0.7352332360639006:4.037704851297755] plot(\x,{(\x)});
\draw[line width=2pt,color=zzttff,smooth,samples=100,domain=-0.7352332360639006:4.037704851297755] plot(\x,{(2*(\x))/3.141592653589793});

\end{axis}
\end{tikzpicture}

        Because $\sin (x)$ is strictly concave on $[0, \pi / 2] .$ We have: $\frac{2}{\pi} x \leq \sin (x) \leq x$ for all $x \in[0, \pi / 2].$ Thus,
        \[
        \frac{(2 / \pi) x}{x^{p}}=\frac{2}{\pi} x^{1-p} \leq \frac{\sin (x)}{x^{p}} \leq \frac{x}{x^{p}}=x^{1-p}
        \]
        on this interval. This is followed by:
        \[
        \frac{2}{\pi} \int_{0}^{\pi / 2} x^{1-p} \mathrm{d} x \leq \int_{0}^{\pi / 2} \frac{\sin (x)}{x^{p}} \mathrm{d} x \leq \int_{0}^{\pi / 2} x^{1-p} \mathrm{d} x
        \]
        For $p \geq 2, \int_{0}^{\pi / 2} x^{1-p} \mathrm{d} x$ diverges, and therefore $\int_{0}^{\pi / 2} \frac{\sin (x)}{x^{p}} \mathrm{d} x$ diverges. For $p<2$, we have the integral being finite. 
        % \[
        % \frac{(\pi / 2)^{1-p}}{2-p} \leq \int_{0}^{\pi / 2} \frac{\sin (x)}{x^{p}} \mathrm{d} x \leq \frac{(\pi / 2)^{2-p}}{2-p}
        % \]
        % and thus the integral is finite. 
    \end{itemize}
    
   % \begin{thm}[Limit Comparison test]. Let $b$ be a real number or $b=\infty,$ let $a<b$. Let $f$ and $g$ be functions that are continuous on $[a, b),$ let $f \geq 0$ there.\\ 
        %     Assume that the limit $\lim _{x \rightarrow b^{-}}\left(\frac{f(x)}{g(x)}\right)$ exists finite, but is not equal to zero. Then the integral $\int_{a}^{b} f(x) d x$ converges if and only if the integral $\int_{a}^{b} g(x) d x$ converges.
        % \end{thm}

    
\end{numedquestion}

\end{document}