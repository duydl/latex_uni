\documentclass[11pt]{exam}
\usepackage{amsmath, amsthm}  
\usepackage{enumitem}
\usepackage[T1]{fontenc}
\usepackage[bitstream-charter]{mathdesign}
\usepackage{parskip}
\usepackage{pgfplots}
\usepackage{booktabs}
\usepackage{tabularx}
\pgfplotsset{soldot/.style={color=blue,only marks,mark=*}} \pgfplotsset{holdot/.style={color=blue,fill=white,only marks,mark=*}}
\pgfplotsset{compat=1.17} 
% Theorems
\theoremstyle{definition}
\newtheorem*{defi}{Definition}
\newtheorem*{thm}{Theorem}

% Special sets
\newcommand{\C}{\mathbb{C}}
\newcommand{\CP}{\mathbb{CP}}
\newcommand{\GG}{\mathbb{G}}
\newcommand{\N}{\mathbb{N}}
\newcommand{\Q}{\mathbb{Q}}
\newcommand{\R}{\mathbb{R}}
\newcommand{\RP}{\mathbb{RP}}
\newcommand{\T}{\mathbb{T}}
\newcommand{\Z}{\mathbb{Z}}
\renewcommand{\H}{\mathbb{H}}

\pagenumbering{gobble}
\newcounter{questionCounter}
\newenvironment{numedquestion}[0]{%
	\stepcounter{questionCounter}%
    \vspace{.2in}%
        \ifx\writtensection\undefined
        \noindent{\bf \questiontype \; \arabic{questionCounter}. }%
        \else
          \if\writtensection
          \noindent{\bf \questiontype \; \arabic{questionCounter}. }%
          \else
          \noindent{\bf \questiontype \; \writtensection.\arabic{questionCounter} }%
        \fi
    \vspace{0.3em} \hrule \vspace{.1in}%
}{}



\footer{}{\thepage}{}
\renewcommand{\frac}{\dfrac}

\usepackage{multicol}

\newcommand{\myclass}{Calculus with Exercises A}
\newcommand{\myname}{DO LE DUY}
\newcommand{\myhwtype}{Report 12}
\newcommand{\questiontype}{Problem}
\newcommand{\writtensection}{12}
\newtheorem*{prop}{Proposition}
\usepackage{pgf,tikz,pgfplots}
\pgfplotsset{compat=1.15}
\usepackage{mathrsfs}
\usetikzlibrary{arrows}
\headrule
\header{{\myname}}%
{\emph{\myclass}}%
{\textbf{\myhwtype }}

\begin{document}
\thispagestyle{empty}
\begin{center}
  {\Large \textbf{\myclass{} — \myhwtype{} }} \\
  {\myname{} } \\
  \today
\end{center}

\begin{numedquestion}
    \begin{align*}
        \sum_{k=1}^{\infty} \frac{1}{k(k+2)(k+4)} &= \sum_{k=1}^{\infty}\frac{1}{8} \left(\frac{1}{k} - \frac{1}{k+2} - \frac{1}{k+2} + \frac{1}{k+4} \right)\\
        &= \frac{1}{8} \left(\sum_{k=1}^{\infty}\left( \frac{1}{k} - \frac{1}{k+2}\right) - \sum_{k=1}^{\infty}\left( \frac{1}{k+2} - \frac{1}{k+4} \right) \right)\\
        &= \frac{1}{8} \left( 1 + \frac{1}{2} - \frac{1}{3} - \frac{1}{4} \right)\\
        &= \frac{11}{96}
    \end{align*}
    
    
\end{numedquestion} 

\begin{numedquestion}
    Because $\sum_{k=1}^{\infty} a_{k}$ is a convergent series of positive terms, $a_k$ will go to zero as $k$ go to $\infty$. Therefore, there exist a $h$ such that for all $a_k$ which $k>h$, $a_k < 1$. We have $\underbrace{\sum_{k=1}^{\infty} a_{k}}_\text{converges} = \underbrace{\sum_{k=1}^{h} a_{k}}_\text{finite} + \underbrace{\sum_{k=h+1}^{\infty} a_{k}}_\text{converges}$. But as $a_k^2 < a_{k}$ for all $k > h$, $\sum_{k=h+1}^{\infty} (a_{k})^2$ must be smaller than $\sum_{k=h+1}^{\infty} a_{k}$ and converge (Direct Comparison Test). We could conclude that $\underbrace{\sum_{k=1}^{h} a_{k}^2}_\text{finite} + \underbrace{\sum_{k=h+1}^{\infty} a_{k}^2}_\text{converges} = \underbrace{\sum_{k=1}^{\infty} a_{k}^2}_\text{converges}$. 

    The converse does not hold. For example, $\sum_{k=1}^{\infty} \frac{1}{n^2}$ converges, but $\sum_{k=1}^{\infty} \frac{1}{n}$ does not.    
    
\end{numedquestion}

\begin{numedquestion}
(1) $\sum_{n=2}^{\infty} \frac{1}{n^{2}(\log n)}$ 

Using comparison test with $\frac{1}{n^2}$, we have $\frac{1}{n^{2}(\log n)} < \frac{1}{n^2}$ for all $n > 1$. But $\frac{1}{n^2}$ converges, therefore $\sum_{n=2}^{\infty} \frac{1}{n^{2}(\log n)}$  also converges.

(2) $\sum_{n=1}^{\infty} \frac{2^{n} n !}{n^{n}}$ .

Using Cauchy’s root test, we have $\lim_{n \rightarrow \infty} \frac{2^{n+1} (n+1) !}{(n+1)^{n+1}}\frac{n^{n}}{2^{n} n !}  = \lim_{n \rightarrow \infty}\frac{2n^{n}}{(n+1)^n}$ = $\frac{2}{e}$ < 1. Therefore, the series converges.  

(3) $\sum_{n=1}^{\infty} \frac{1}{\sqrt{1+n^{3}}}$ 

Using comparison test with $\frac{1}{n^{3/2}}$, we have $\frac{1}{\sqrt{1+n^{3}}} < \frac{1}{n^{3/2}}$ for all $n > 1$. But $\sum_{n=1}^{\infty} \frac{1}{n^{3/2}}$ converges, therefore $\sum_{n=1}^{\infty} \frac{1}{\sqrt{1+n^{3}}}$  also converges.

(4) $\sum_{n=1}^{\infty}(-1)^{n-1} \frac{n}{n^{2}+1}$ 

Applying alternating series test, because $\frac{n}{n^{2}+1}$ decrease monotonically to zero, we could conclude that the series converges. 
\end{numedquestion}

\begin{numedquestion}
The reason for the value $s_{n}$ stops changing after some $n$ is that there is maximum memory the program allows for a variable. When $\frac{1}{n}$ is smaller than the precision of the program, it is rounded to zero, then $s_{n}$ will stop changing. 

Nevertheless, it turns out to be really difficult to calculate the final sum. In C for example, we will have to specify the floating points before printing, so it is impossible to monitor the sum. In Python, the program runs really slow it is quite hard to wait till the end. 

C Program:
\begin{verbatim}
#include <stdio.h>
void main()
{
    float number = 1 , sum = 0, i = 1;
    while (number > 0)
    {   number = 1 / i;
        sum = sum + number;
        if (number == 0){ \\ could take a really long time for this to be True
            break;
        }
        i = i + 1;       
    }
    printf("\n The sum till converging is %.5lf", sum); 
    \\ have to define precision here
}
\end{verbatim}
Python Program: 
\begin{verbatim}
sum = 0
i = 1
while True:
    sum = sum + 1/i
    i = i + 1
    if 1/i == 0: 
        break
print(sum)
\end{verbatim}

\emph{Or I may misunderstand the phenomenon entirely. }
\end{numedquestion}

\end{document}