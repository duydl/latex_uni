\documentclass[11pt]{exam}
\usepackage{amsmath, amsthm}  
\usepackage{enumitem}
\usepackage[T1]{fontenc}
\usepackage[bitstream-charter]{mathdesign}
\usepackage{parskip}
\usepackage{pgfplots}
\usepackage{booktabs}
\usepackage{tabularx}
\pgfplotsset{soldot/.style={color=blue,only marks,mark=*}} \pgfplotsset{holdot/.style={color=blue,fill=white,only marks,mark=*}}
\pgfplotsset{compat=1.17} 
% Theorems
\theoremstyle{definition}
\newtheorem*{defi}{Definition}
\newtheorem*{thm}{Theorem}

% Special sets
\newcommand{\C}{\mathbb{C}}
\newcommand{\CP}{\mathbb{CP}}
\newcommand{\GG}{\mathbb{G}}
\newcommand{\N}{\mathbb{N}}
\newcommand{\Q}{\mathbb{Q}}
\newcommand{\R}{\mathbb{R}}
\newcommand{\RP}{\mathbb{RP}}
\newcommand{\T}{\mathbb{T}}
\newcommand{\Z}{\mathbb{Z}}
\renewcommand{\H}{\mathbb{H}}

\pagenumbering{gobble}
\newcounter{questionCounter}
\newenvironment{numedquestion}[0]{%
	\stepcounter{questionCounter}%
    \vspace{.2in}%
        \ifx\writtensection\undefined
        \noindent{\bf \questiontype \; \arabic{questionCounter}. }%
        \else
          \if\writtensection
          \noindent{\bf \questiontype \; \arabic{questionCounter}. }%
          \else
          \noindent{\bf \questiontype \; \writtensection.\arabic{questionCounter} }%
        \fi
    \vspace{0.3em} \hrule \vspace{.1in}%
}{}



\footer{}{\thepage}{}
\renewcommand{\frac}{\dfrac}

\usepackage{multicol}

\newcommand{\myclass}{Calculus with Exercises A}
\newcommand{\myname}{DO LE DUY}
\newcommand{\myhwtype}{Report 10}
\newcommand{\questiontype}{Problem}
\newcommand{\writtensection}{10}

\headrule
\header{{\myname}}%
{\emph{\myclass}}%
{\textbf{\myhwtype }}

\begin{document}
\thispagestyle{empty}
\begin{center}
  {\Large \textbf{\myclass{} — \myhwtype{} }} \\
  {\myname{} } \\
  \today
\end{center}
\begin{numedquestion}
    \begin{enumerate}[label = {(\arabic*)}]
        \item 
        \begin{flalign*}
            \int e^{\sqrt{x}} d x &= \int 2 \sqrt{x} e^{\sqrt{x}} d(\sqrt{x}) \\
            \text {Set } \sqrt{x} &= t, \text{ then}: \\
            \int e^{\sqrt{x}} d x &= \int 2 t e^t d t \\
            &= 2 t e^t - 2 \int e^t dt\\
            &= 2t e^t - 2 e^t \\
            &= 2 e^{\sqrt{x}} (\sqrt{x}  - 1)
        && \end{flalign*}
        \item
        \begin{flalign*}
            \int \frac{1}{\sqrt{\sqrt{x}+1}} d x & \text{, set } u = \sqrt{x} +1. \\       
            \text{Then } x = u^2 - 2u + 1 & \text{, and } d x = 2( u - 1) d u \text{. We have:} \\
            \int \frac{1}{\sqrt{\sqrt{x}+1}} d x &= 2 \int \frac{1}{\sqrt{u}} (u - 1) d u \\
            &= 2 \int u^{1/2} du - 2 \int u^{-1/2} du \\
            &= \frac{4}{3} u^{3/2} - 4 u^{1/2} \\
            &= \frac{4}{3} (\sqrt{x} - 2) (\sqrt{x} + 1)^{1/2} + C
        && \end{flalign*}
        \item
        \begin{flalign*}
            \int \sqrt{1-x^{2}} d x, & \text{ set } x = \sin \theta.\\
            \text{Then } \sqrt{1 - x^2} = \cos \theta, & \text{and } dx = \cos \theta d \theta. \text{ We have:}\\
            \int \sqrt{1-x^{2}} d x &= \int (\cos \theta)^2 d \theta \\
            &= \int \frac{\cos (2 \theta) + 1}{2} d \theta = \frac{\sin {2 \theta}}{4} + \frac{\theta}{2} + C\\
            &= \frac{1}{2} \left( x \sqrt{1- x^2} + \arcsin x \right) + C
        && \end{flalign*}
        \item
        \begin{flalign*}
            \int x(\ln x)^{2} d x &= \int (\ln x) (x\ln x dx), \text{set } u = \ln x \text{ and } dv =  (x\ln x dx) \\
            \text{We will find } v: v &= \int x \ln x d x \\
            &= \frac{x^2}{2} \ln x - \frac{1}{2} \int x d x \\
            &= \frac{x^2}{2} \left(\ln x - \frac{1}{2}\right),\text{ therefore: }\\
            \int x(\ln x)^{2} d x &= \ln x \left(\frac{x^2}{2} \left(\ln x - \frac{1}{2}\right)\right) - \int \frac{1}{x} \frac{x^2}{2}\left(\ln x - \frac{1}{2}\right) d x\\
            & = \ln x \left(\frac{x^2}{2} \left(\ln x - \frac{1}{2}\right)\right) - \int \frac{x}{2}\left(\ln x - \frac{1}{2}\right) d x \\
            &= \frac{x^2}{2} (\ln x)^2 - \frac{x^2}{4}\ln x - \frac{1}{2} \frac{x^2}{2} \left(\ln x - \frac{1}{2}\right) + C\\
            &= \frac{x^2}{2} (\ln x)^2 - \frac{x^2}{2}\ln x - \frac{x^2}{4} + C
        && \end{flalign*}
        \item
        \begin{flalign*}
            \text{We will prove} & \text{ a general reduction formula:}\\
            \int \sec ^{n} x d x &= \frac{\sec ^{n-2} x \tan x}{n-1}+\frac{n-2}{n-1} \int \sec ^{n-2} x d x, n \neq 1. \text{ We have:} \\
            \int \sec ^{n} x d x &= \int \sec ^{n-2} (\tan^2 x + 1) d x = \int \sec^{n-2} x + \int \tan x \sec^{n-3} d (\sec x) \\
            &= \int \sec^{n-2} x + \frac{\sec^{n-2} x \tan x}{n-2} - \frac{1}{n-2} \int \sec^{n-2} \sec^2 x d x. \text{ Therefore: }\\
            \int \sec ^{n} x d x &= \frac{\sec ^{n-2} x \tan x}{n-1}+\frac{n-2}{n-1} \int \sec ^{n-2} x d x. \text{ Thus, we have:} \\
            \int \sec ^{3} x d x &= \frac{\sec x \tan x}{2} + \frac{1}{2} \int \sec x dx = \frac{1}{2}\sec x \tan x + \frac{1}{2}\ln |\tan x + \sec x| \\
        && \end{flalign*}
        \item
        \begin{flalign*}
            \int \frac{1}{\sqrt{x^{2}-1}} d x, &\text{ set } x = \sec \theta. \text{ So } d x = \tan \theta \sec \theta d \theta, \text{ we have: } \\
            \int \frac{1}{\sqrt{x^{2}-1}} d x &= \int \frac{1}{\sqrt{(\sec \theta)^2 -1}} \tan \theta \sec \theta d \theta \\
            &= \int \sec \theta d \theta = \ln \left|\tan \theta + \sec \theta \right| + C \\
            &= \ln |\sqrt{x^2 -1} + x| + C
        && \end{flalign*}
        \item
        \begin{flalign*}
            \int \left(\sqrt{x^{2}+1}\right) d x, &\text{ set } x = \tan \theta. \text{ So } d x =  (\sec \theta)^2 d \theta, \text{ we have: } \\  
            \int \left(\sqrt{x^{2}+1}\right) d x &= \int \sqrt{(\tan \theta)^2 + 1} \sec ^2 \theta  d\theta \\
            &= \int \sec^3 \theta d \theta \\
            &= \frac{1}{2}\sec \theta \tan \theta + \frac{1}{2}\ln |\tan \theta + \sec \theta| \text{ from (5)}   \\ 
            &= \frac{1}{2}x \sqrt{x^2 +1} + \frac{1}{2}\ln |x + \sqrt{x^2+1}|  
            && \end{flalign*}
        \item
        \begin{flalign*}
            \int \frac{2 x^{2}+x+1}{(x+3)(x-1)^{2}} d x &= \int \left(\frac{1}{(x-1)^2} + \frac{1}{x - 1} + \frac{1}{x+3}\right) dx \\
            &= \frac{-1}{x-1} + \ln (x-1) + \ln (x+3)
        && \end{flalign*}
    \end{enumerate}
\end{numedquestion}
\end{document}