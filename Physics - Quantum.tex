\documentclass[11pt]{exam}
\usepackage{amsmath, amsthm}  
\usepackage{enumitem}
\usepackage[T1]{fontenc}
\usepackage[bitstream-charter]{mathdesign}
\usepackage{parskip}
\usepackage{pgfplots}
\usepackage{booktabs}
\usepackage{tabularx}
\pgfplotsset{soldot/.style={color=blue,only marks,mark=*}} \pgfplotsset{holdot/.style={color=blue,fill=white,only marks,mark=*}}
\pgfplotsset{compat=1.17} 
% Theorems
\theoremstyle{definition}
\newtheorem*{defi}{Definition}
\newtheorem*{thm}{Theorem}

% Special sets
\newcommand{\C}{\mathbb{C}}
\newcommand{\CP}{\mathbb{CP}}
\newcommand{\GG}{\mathbb{G}}
\newcommand{\N}{\mathbb{N}}
\newcommand{\Q}{\mathbb{Q}}
\newcommand{\R}{\mathbb{R}}
\newcommand{\RP}{\mathbb{RP}}
\newcommand{\T}{\mathbb{T}}
\newcommand{\Z}{\mathbb{Z}}
\renewcommand{\H}{\mathbb{H}}

\pagenumbering{gobble}
\newcounter{questionCounter}
\newenvironment{numedquestion}[0]{%
	\stepcounter{questionCounter}%
    \vspace{.2in}%
        \ifx\writtensection\undefined
        \noindent{\bf \questiontype \; \arabic{questionCounter}. }%
        \else
          \if\writtensection
          \noindent{\bf \questiontype \; \arabic{questionCounter}. }%
          \else
          \noindent{\bf \questiontype \; \writtensection.\arabic{questionCounter} }%
        \fi
    \vspace{0.3em} \hrule \vspace{.1in}%
}{}



\footer{}{\thepage}{}
\renewcommand{\frac}{\dfrac}

\usepackage{multicol}


\begin{document}
\noindent  
\large \textbf{ILAS Seminar — E2: Quantum Physics} \smallskip \newline
Assignment 1 \bigskip \newline
\small Written on \today \medskip
\hrule
\textbf{Photoelectric Effect} is a phenomenon that describes the ejection of electron  from a substance surface upon exposure to electromagnetic radiation. In classical wave theory, as light is considered a wave with the energy distributed evenly, it is expected that when using very dim light, it would take some time for enough energy to build up to eject an electron. However, experiments show that if light of a certain frequency can eject electrons, there is never a time delay no matter how dim the light is. Furthermore, if the frequency is below a threshold, no electrons would be emitted, regardless of the light intensity or time exposure. 

In 1905, using the idea first put forward by Max Planck that energy comes in packets (quanta), Albert Einstein explained the phenomenon by proposing that light must be a stream of discrete packets of energy. Each packet of light energy is called a photon and each photon has an energy of $hf$, which the Planck's constant times the frequency of the light. Energy of light is concentrated in the photons, rather than evenly distributed along the wave. A dimmer light means fewer photons, but energy of individual photon with the similar frequency is the same. So for a specific frequency light, an electron will be ejected immediately after exposure to light as long as there is a single photon that has enough energy.
\end{document}