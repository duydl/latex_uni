% DO LE DUY
% 1026-32-2038
% July 14 2020

\documentclass[11pt]{article}
\usepackage[T1]{fontenc}
\usepackage[bitstream-charter]{mathdesign}
\usepackage{amsmath, amsthm} 
\usepackage{geometry}
\geometry{left=2cm,right=2cm,top=1cm,bottom=2cm,includeheadfoot}
\title{Practice of Basic Informatics - Homework 9}
\author{by Do Le Duy}
\date{\today}

\begin{document}
\maketitle
\section{Problem}
Three equations:
\begin{align*}
    &\text{Sum}=\sum_{i=1}^{5} x_{i} \\
    &\text { Arithmetic Mean }=\frac{1}{5} \sum_{i=1}^{n} x_{i} \\
    &\text { Geometric Mean }=\sqrt[5]{x_{1} \times x_{2} \times x_{3} \times x_{4} \times x_{5}}   
\end{align*}
FORTRAN Program to solve the problem:
\begin{verbatim} 
    program calculate_three_values

    implicit none
    REAL               :: thesum = 0, themultiple = 1, a
    INTEGER            :: i, n
    write(*,*) 'Please, input the number of sample n'
    read(*,*) n 

    if(n < 2 .or. n > 50) then
        write(*,*) 'Sample data must be between 2 and 50'
        stop
    end if

    do i = 1, n
        write(*,*) 'Please, input the sample', i
        read(*,*) a
        thesum = thesum + a 
        themultiple = themultiple * a
    end do

    write(*,*) 'Sum: ', thesum
    write(*,*) 'Arithmetic Mean: ', thesum/dble(n)
    write(*,*) 'Geometric Mean: ', themultiple**(1/dble(n))
    end program calculate_three_values
\end{verbatim} 

\section{Answers by the program}
\subsection{Set 1}
This is the programming running for Set 1:

\begin{verbatim}
Please, input the number of sample n
4
 Please, input the sample           1
1
 Please, input the sample           2
5
 Please, input the sample           3
10
 Please, input the sample           4
11
 Sum:    27.0000000    
 Arithmetic Mean:    6.7500000000000000     
 Geometric Mean:    4.8427346405845064     
    
\end{verbatim}

\subsection{Set 2}
This is the programming running for Set 1:

\begin{verbatim}
    Please, input the number of sample n
    9
     Please, input the sample           1
    57.9
     Please, input the sample           2
    655.7
     Please, input the sample           3
    44.3
     Please, input the sample           4
    1000.5
     Please, input the sample           5
    3.5
     Please, input the sample           6
    6.4
     Please, input the sample           7
    5.3
     Please, input the sample           8
    78.6
     Please, input the sample           9
    12.5
     Sum:    1864.70007    
     Arithmetic Mean:    207.18889702690973     
     Geometric Mean:    38.734296233169147                    
\end{verbatim}  

\subsection{Error Message}
This is the programming running for Error Message:

\begin{verbatim}
    Please, input the number of sample n
    1
    Sample data must be between 2 and 50           
\end{verbatim} 

\section{FORTRAN programming}
\subsection{What I have learned today}
\begin{itemize}
    \item Using loop and conditional in FORTRAN.
\end{itemize}
\subsection{Difficult points}
\begin{itemize}
    \item I am trying to figure out how to write without space. 
\end{itemize}
\end{document}