\documentclass{article}
\usepackage{textcomp} %<-- enables use of various symbols(you can find other symbols in  http://tug.ctan.org/info/symbols/comprehensive/symbols-a4.pdf)
\title{Practice in Chapter 6} % with \maketitle, title is generated
\author{Taro Kyodai} % with \maketitle, autthor is generated

\begin{document}


\maketitle % title, author, date of issue are generated. date of issue is automatically generated unless you define it.

\tableofcontents % the command generate table of contents in the LaTeX document. to generate latest state, you may need to compile it more than two times.

\section{6.2 Line breaking}

Text\\ % Starts a new line withouy starting a new paragraph
Text\newline % Same as \\
Text\\* % additionally prohibitis a page break after the forced line break
A new page starts\newpage %starts a new page

\section{6.3 Special characters and symbols}

``Please press the 'x' key.''

''Please press the 'x' key.''

"Please press the 'x' key."

% Note that we need to pay attention to the direction of double quotations
% Emacs will automatically replace the ''symbol for te correct quotation marks `` or '' as corresponds. you will experience one more advantage of using emacs.

daughter-in-law, X-rated\\
pages 13--67\\
yes---or no? \\
$0$, $1$, and $-1$ \\ %<-- to give a space between numbers, $ is necessary
0,1 and -1
%<-- An empty line starts a new line

http://server/\~{}username/filename\\ 
http://server/~username/filename\\ %<-- ~ is a special character. see 5.7
http://server/$\sim$username/filename

30 \textcelsius{} is 86 \textdegree{}F.

Not like this ... but like this:\\
New York, Tokyo, Budapest, \ldots
% you will find slight differece of spasing in dots
% \cdots similar with \ldots is used in next chapter

\section{6.4 Footnotes}

% footnote is required in the following homework.
% please master how to use it

Footnotes\footnote{This is a foot note.} are often used
by people using \LaTeX. %<-- you can find it on footer of the page

Footnotes\footnote{This is a foot note.} are often used
by people using \LaTeX. %<-- you can find it on footer of the page


\section{6.5 Emphasized words}

This is an example of an \emph{emphasized text}.\\ %<-- emphasized example
This is an example of a \underline{underlined text}.\\%<-- underline example

\emph{If you use emphasizing inside a piece of emphasized text, then \LaTeX{} uses the \emph{normal} font for emphasizing.} %<-- \emph is contextual. effect of emphasized text can be deleted.

\textbf{\textsl{boltface slanted text}} %<-- you can give double emphasized effect in this way.

\section{6.6 Titles, Chapters, and Sections}

\section{This is a section}
\section*{This is another section}
\subsection{This is a subsection}
\subsubsection{This is a subsubsection}
\paragraph{this is a paragraph}
\subparagraph{this is a subparagraph}

\section{Help} % numbered section appears on table of contents
\section*{Help} % non-numbered section appears but not on table of contents

\end{document}
