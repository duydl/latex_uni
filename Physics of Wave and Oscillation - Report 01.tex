\documentclass[11pt]{exam}
\newcommand{\myclass}{Physics of Wave and Oscillation}
\newcommand{\myname}{DO LE DUY}
\newcommand{\myhwtype}{Report 01}
% Prefix for numedquestion's

\input{_header}

\begin{document}
\thispagestyle{plain}
\begin{center}
  {\Large \textbf{\myclass{} - \myhwtype{} }}\\
  \vspace{0.6em} {\myname{} (\small{ID: \mystudentid{}})} \\
  \today
\end{center}


\textbf{Prove Euler's formula on a complex plane}
\vspace{0.3em} \hrule \vspace{.1in}

Euler's formula states that for any real number \(x\) :
\[
e^{i x}=\cos x+i \sin x
\]
where \(e\) is the base of the natural logarithm, \(i\) is the imaginary unit, with \(x\) given in radians. \newline
As all complex numbers can be expressed in polar coordinates, there exist some \(r\) and \(\theta\) depending on \(x\) such that:
\[
e^{i x}=r(\cos \theta+i \sin \theta)
\]
From any of the definitions of the exponential function we know that the derivative of \(e^{i x}\) is \(i e^{i x}\). Therefore,
differentiating both sides gives:
\[
i e^{i x}=(\cos \theta+i \sin \theta) \frac{d r}{d x}+r(-\sin \theta+i \cos \theta) \frac{d \theta}{d x}
\]
\[
\iff i r(\cos \theta+i \sin \theta)=(\cos \theta+i \sin \theta) \frac{d r}{d x}+r(-\sin \theta+i \cos \theta) \frac{d \theta}{d x}
\]
As real and imaginary parts on two sides are equal each other respectively, we have \(\frac{d r}{d x}=0\) and \(\frac{d \theta}{d x}=1 .\) Thus, \(r\) is a constant, and \(\theta\) is \(x+C\) where \(C\) is a constant. Let x = 0, we have \(r(0)=1\) and therefore \(\sin C = 0\) and \(r\cos C = 1\), giving \(r=1\) and \(\theta=x .\) We have proved the formula:
\[
e^{i x}=1(\cos x+i \sin x)=\cos x+i \sin x
\]
\end{document}