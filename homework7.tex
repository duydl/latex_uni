% DO LE DUY
% 1026-32-2038
% June 30 2020


\documentclass[11pt]{article}
\DeclareUnicodeCharacter{2212}{-}
\usepackage{amsmath}
% \usepackage[showframe]{geometry}
\title{The Enigmatic Number $e$ \thanks{This text was originally cited from Sarah Glaz (University of Connecticut), "The Enigmatic Number \(e:\) A History in Verse and Its Uses in the Mathematics Classroom - The Annotated Poem," Convergence (November 2010 ), DOI:10.4169/loci003482}}
\author{by Do Le Duy}
\date{\today}

\begin{document}

\maketitle
\noindent It ambushed Napier at Gartness, \\
like a swashbuckling pirate \\
leaping from the base. \\
He felt its power, but never realized its nature. \\
$e$’s first appearance in disguisea tabular array \\
of values of $ln$, was logged in an appendix \\
to Napier’s posthumous publication. \\
Oughtred, inventor of the circular slide rule, \\
still ignorant of $e$’s true role, \\
performed the calculations. \\

\noindent A hundred thirteen years the hit and run goes on. \\
There and not there — elusive $e$, \\
escape artist and trickster, \\
weaves in and out of minds and computations: \\

\noindent Saint-Vincent caught a glimpse of it under rectangular hyperbolas; \\
Huygens mistook its rising trace for logarithmic curve; \\
Nicolaus Mercator described its log as natural \\
without accounting for its base; \\
Jacob Bernoulli, compounding interest continuously, \\
came close, yet failed to recognize its face; \\
and Leibniz grasped it hiding in the maze of calculus, \\
natural basis for comprehending change — but \\
misidentified as $b$. \\

\noindent The name was first recorded in a letter \\
Euler sent Goldbach in November 1731: \\
\emph{“$e$ denontat hic numerum, cujus logarithmus hyperbolicus est=1.”} \\
Since $a$ was taken, and Euler \\
was partial to vowels, \\
$e$ rushed to make a claim — the next in line.\\

\noindent We sometimes call $e$ Euler’s Number: he knew\\
$e$ in its infancy as $2.718281828459045235$. \\

\noindent On Wednesday, 6th of May, 2009,\\
$e$ revealed itself to Kondo and Pagliarulo,\\
digit by digit, to $200,000,000,000$ decimal places.\\
It found a new digital game to play.\\

\noindent In retrospect, following Euler’s naming,\\
$e$ lifted its black mask and showed its limit:
\begin{flalign*} \hspace{1cm} e = \lim_{n \rightarrow \infty} (1 + \frac{1}{n})n&&\end{flalign*}
Bernoulli’s compounded interest for an investment of one.\\

\noindent Its reciprocal gave Bernoulli many trials,\\
from gambling at the slot machines to deranged parties\\
where nameless gentlemen check hats with butlers at the door,\\
and when they leave, $e$’s reciprocal hands each a stranger’s hat.\\

\noindent In gratitude to Euler, $e$ showed a serious side,\\
infinite sum representation:
\begin{flalign*} \hspace{1cm} e=\sum_{n=0}^{\infty} \frac{1}{n !}=\frac{1}{0 !}+\frac{1}{1 !}+\frac{1}{2 !}+\frac{1}{3 !}+\cdots&&\end{flalign*}
For Euler’s eyes alone, $e$ fanned the peacock tail of\\
$\frac{e−1}{2}$’s continued fraction expansion,\\
displaying patterns that confirmed\\
its own irrationality.\\
A century has passed till $e $— through Hermite’s pen,\\
was proved to be a transcendental number.\\
But to this day it teases us with\\
speculations about $e^e$\\

\noindent $e$’s abstract beauty casts a glow on Euler’s Identity:
\begin{flalign*} \hspace{1cm} e^{i\pi} + 1 = 0&&\end{flalign*}
the elegant, mysterious equation,\\
where waltzing arm in arm with $i$ and $\pi$,\\
$e$ flirts with complex numbers and roots of unity. \\

\noindent We meet $e$ nowadays in functional high places\\
of Calculus, Differential Equations, Probability, Number Theory,\\
and other ancient realms:
\begin{flalign*} \hspace{1cm} y = e^x&&\end{flalign*}
$e$ is the base of the unique exponential function\\
whose derivative is equal to itself.\\
The more things change the more they stay the same.\\
$e$ gathers gravitas as solid under integration,
\begin{flalign*} \hspace{1cm} \int e^x d x = e^x + c&&\end{flalign*}
a constant $c$, is the mere difference;\\
and often $e$ makes guest appearances in Taylor series expansions.\\
And now and then $e$ stars in published poetry —\\
honors and administrative duties multiply with age.\\
\section*{3 Mathematical Formulas in Latex}
\subsection*{3.1 What I have learned today}
Using \emph{thanks} command in the title. 
\subsection*{3.2 Difficult points}
I receive the error : \emph{"Package inputenc: Unicode character − (U+2212) (inputenc)	not set up for use with LaTeX.} The fix is to add \emph{\\DeclareUnicodeCharacter\{2212\}\{-\}} at the preamble or to compile with a different compiler. But I did not understand the reason of the error.

\end{document}