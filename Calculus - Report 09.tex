\documentclass[11pt]{exam}
\usepackage{amsmath, amsthm}  
\usepackage{enumitem}
\usepackage[T1]{fontenc}
\usepackage[bitstream-charter]{mathdesign}
\usepackage{parskip}
\usepackage{pgfplots}
\usepackage{booktabs}
\usepackage{tabularx}
\pgfplotsset{soldot/.style={color=blue,only marks,mark=*}} \pgfplotsset{holdot/.style={color=blue,fill=white,only marks,mark=*}}
\pgfplotsset{compat=1.17} 
% Theorems
\theoremstyle{definition}
\newtheorem*{defi}{Definition}
\newtheorem*{thm}{Theorem}

% Special sets
\newcommand{\C}{\mathbb{C}}
\newcommand{\CP}{\mathbb{CP}}
\newcommand{\GG}{\mathbb{G}}
\newcommand{\N}{\mathbb{N}}
\newcommand{\Q}{\mathbb{Q}}
\newcommand{\R}{\mathbb{R}}
\newcommand{\RP}{\mathbb{RP}}
\newcommand{\T}{\mathbb{T}}
\newcommand{\Z}{\mathbb{Z}}
\renewcommand{\H}{\mathbb{H}}

\pagenumbering{gobble}
\newcounter{questionCounter}
\newenvironment{numedquestion}[0]{%
	\stepcounter{questionCounter}%
    \vspace{.2in}%
        \ifx\writtensection\undefined
        \noindent{\bf \questiontype \; \arabic{questionCounter}. }%
        \else
          \if\writtensection
          \noindent{\bf \questiontype \; \arabic{questionCounter}. }%
          \else
          \noindent{\bf \questiontype \; \writtensection.\arabic{questionCounter} }%
        \fi
    \vspace{0.3em} \hrule \vspace{.1in}%
}{}



\footer{}{\thepage}{}
\renewcommand{\frac}{\dfrac}

\usepackage{multicol}
\newcommand{\myclass}{Calculus with Exercises A}
\newcommand{\myname}{DO LE DUY}
\newcommand{\myhwtype}{Report 09}
\newcommand{\questiontype}{Problem}
\newcommand{\writtensection}{9}

\headrule
\header{{\myname}}%
{\emph{\myclass}}%
{\textbf{\myhwtype }}
\newtheorem*{prop}{Proposition}
\begin{document}
\thispagestyle{empty}
\begin{center}
  {\Large \textbf{\myclass{} — \myhwtype{} }} \\
  {\myname{} } \\
  \today
\end{center}

\begin{numedquestion}
    \begin{prop}
        If $f$ is continuous on an interval $[a,b]$ and $\int^b_a f(x)g(x) = 0$ for every continuous function $g$ on $[a,b]$, then $f(x)=0$ on $[a,b]$.
    \end{prop} 
    We will use contradiction to prove the above proposition. Assume that there is a continuous function $f(x) \neq 0$ on $[a,b]$ such that $\int ^b _a f(x)g(x) = 0$ for every continuous function $g(x)$. Then for $g(x) = f(x)$ we have $\int ^b _a f(x)g(x)dx = \int ^b _a f(x)^2dx = 0$. But $\int ^b _a f(x)^2dx > 0$ for every continuous function $f(x)$ that is not identically equal zero on $[a,b]$. We have reached a contradiction, thus, the proposition must be true. 
\end{numedquestion}  

\begin{numedquestion}
    \begin{enumerate}[label = {(\arabic*)}]
        \item $\int^{\pi/4}_0 \sin(2x)\sin(3x)dx = \frac{1}{2}\int^{\pi/4}_0 (\cos x - \cos 5x)dx = \left[\frac{\sin x}{2} - \frac{\sin 5x}{10}\right] \Big|_0^{\pi/4} = \frac{1}{2} - \frac{1}{10}  = \frac{2}{5}$ 
        \item $\int^{\pi/2}_0 \frac{\sin x dx}{2 + \cos x} = -\int^{\pi/2}_{0} \frac{d(\cos x)}{2 +\cos x} = -\ln (2+ \cos x) \Big|_0^{\pi/2}= \ln \frac{3}{2} $
    \end{enumerate}
\end{numedquestion}

\begin{numedquestion}
    \begin{thm}Let \(f\) be a continuous function on an interval \([a, b]\). Then
    \[
    \int_{a}^{b} f(x) d x=\lim _{n \rightarrow \infty} \frac{b-a}{n} \sum_{k=0}^{n-1} f\left(a+\frac{k}{n}(b-a)\right)
    \]
    \end{thm}
    Using the theorem, we have:
    \begin{align*}
        &\lim _{n \rightarrow \infty} \frac{2^{1/n} + 2^{2/n} + \dots + 2^{n-1/n} + 2^{n/n}}{n} \\
        =&\lim _{n \rightarrow \infty} \frac{2^{0/n} + 2^{2/n} + \dots + 2^{n-2/n} + 2^{n-1/n}}{n} \\
        =& \lim_{n \rightarrow \infty} \frac{1-0}{n} \sum_{k = 0}^{n-1} 2^{0 + (k/n)(1-0)} \\
        =&  \int_0^1 2^{x} dx = \frac{2^x}{\ln 2} \Big | _0^1 = \frac{1}{\ln 2}
    \end{align*}
\end{numedquestion}
% Wikipedia -- mean value theorem for definite integrals.
\newpage
\begin{numedquestion}
    For the function $f$, there must exist two number $m$ and $M$ such that $m \leq f(x) \leq M$. We have:
    \[m\int^b_a g(x) dx \leq \int_{a}^{b} f(x) g(x) d x \leq M\int^b_a g(x) dx \]
    Let $\int^b_a g(x) dx = H$, since g is non-negative, we have:
    \[m \leq \frac{1}{H} \int_a^b f(x) g(x) \leq M. \]
    By the intermediate value theorem, function $f$ reaches all value on the interval $[m, M]$, so there exists a $\xi$ on $(a,b)$ that:
    \[\frac{1}{H} \int_a^b f(x) g(x) = f(\xi). \]
    Which means:
    \[\int_{a}^{b} f(x) g(x) d x=f(\xi) \int_{a}^{b} g(x) d x\]

\end{numedquestion}

\end{document}