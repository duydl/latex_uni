\documentclass[11pt]{exam}
\usepackage{amsmath, amsthm}  
\usepackage{enumitem}
\usepackage[T1]{fontenc}
\usepackage[bitstream-charter]{mathdesign}
\usepackage{parskip}
\usepackage{pgfplots}
\usepackage{booktabs}
\usepackage{tabularx}
\pgfplotsset{soldot/.style={color=blue,only marks,mark=*}} \pgfplotsset{holdot/.style={color=blue,fill=white,only marks,mark=*}}
\pgfplotsset{compat=1.17} 
% Theorems
\theoremstyle{definition}
\newtheorem*{defi}{Definition}
\newtheorem*{thm}{Theorem}

% Special sets
\newcommand{\C}{\mathbb{C}}
\newcommand{\CP}{\mathbb{CP}}
\newcommand{\GG}{\mathbb{G}}
\newcommand{\N}{\mathbb{N}}
\newcommand{\Q}{\mathbb{Q}}
\newcommand{\R}{\mathbb{R}}
\newcommand{\RP}{\mathbb{RP}}
\newcommand{\T}{\mathbb{T}}
\newcommand{\Z}{\mathbb{Z}}
\renewcommand{\H}{\mathbb{H}}

\pagenumbering{gobble}
\newcounter{questionCounter}
\newenvironment{numedquestion}[0]{%
	\stepcounter{questionCounter}%
    \vspace{.2in}%
        \ifx\writtensection\undefined
        \noindent{\bf \questiontype \; \arabic{questionCounter}. }%
        \else
          \if\writtensection
          \noindent{\bf \questiontype \; \arabic{questionCounter}. }%
          \else
          \noindent{\bf \questiontype \; \writtensection.\arabic{questionCounter} }%
        \fi
    \vspace{0.3em} \hrule \vspace{.1in}%
}{}



\footer{}{\thepage}{}
\renewcommand{\frac}{\dfrac}

\usepackage{multicol}
\newcommand{\myclass}{Linear Algebra with Exercises A}
\newcommand{\myname}{DO LE DUY}
\newcommand{\myhwtype}{Report 01}
\newcommand{\writtensection}{}
\newcommand{\mystudentid}{1026-32-2038}


\begin{document}
\thispagestyle{plain}
\begin{center}
  {\Large \textbf{\myclass{} - \myhwtype{} }}\\
  \vspace{0.6em} {\myname{} (\small{ID: \mystudentid{}})} \\
  \today
\end{center}


\textbf{Diagonalizing a 2x2 Matrix}
\vspace{0.3em} \hrule \vspace{.1in}

Diagonalize the matrix
\[
A=\begin{bmatrix}
2 & 1 \\
1 & 1
\end{bmatrix}
\]
\begin{enumerate}
    \item The characteristic polynomial of \(A\) is:\newline
    \[p_{A}(x) = \operatorname{det}(A-x I) = 
    \begin{vmatrix}
2 - x & 1 \\
1 & 1 - x   
    \end{vmatrix} =x^2 - 3x +1 \]

    \item Let \(\lambda_{1}<\lambda_{2}\) be the roots of \(P\). We have \newline
    \[\lambda_{1} = \frac{3- \sqrt{5}}{2}\] and \[ \lambda_{2} = \frac{3+ \sqrt{5}}{2}\]


    \item Now we find the eigenspaces \(E_1\) and \(E_2\) corresponding to the two eigenvalues \(\lambda_1\) and \(\lambda_2\) of A. \newline
    \(E_1\) is the null space of the matrix 
    \[ A - \lambda_{1} I = 
    \begin{bmatrix}
    \frac{1 + \sqrt{5}}{2} & 1 \\
    1 & \frac{-1+ \sqrt{5}}{2}
    \end{bmatrix}
    \] 
    and
    \[\begin{bmatrix}
    \frac{1 + \sqrt{5}}{2} & 1 \\
    1 & \frac{-1+ \sqrt{5}}{2}
    \end{bmatrix}
    \begin{bmatrix}
    x \\
    y
    \end{bmatrix} = 0
    \] where \(\begin{bmatrix} x \\ y \end{bmatrix} \) is a vector in \(E_1\). We will use elimination to find the null space: 
    \[ \begin{bmatrix}
    \frac{1 + \sqrt{5}}{2} & 1 \\
    1 & \frac{-1+ \sqrt{5}}{2}
    \end{bmatrix} 
    \rightarrow 
    \begin{bmatrix}
    \frac{1 + \sqrt{5}}{2} & 1 \\
    0 & 0
    \end{bmatrix}
    \]
    Thus, we have \[\frac{1 + \sqrt{5}}{2}x + y = 0 \iff y = \frac{2}{1 - \sqrt{5}}x \]
    Therefore, 
    \[E_1 = 
    \begin{bmatrix}
    1 \\
    \frac{2}{1 - \sqrt{5}}
    \end{bmatrix}x \]where x is an arbitrary constant. 
    
    \item Similarly, we find \(E_2\):
    \[E_2 = 
    \begin{bmatrix}
    1 \\
    \frac{2}{1 + \sqrt{5}}
    \end{bmatrix}x \]
    
    \item To show \(E_1\) and \(E_2\) are orthogonal, we will show that their dot product is equal zero: \[ 
    \begin{bmatrix}
    1 \\
    \frac{2}{1 - \sqrt{5}}
    \end{bmatrix}^{T}
    \begin{bmatrix}
    1 \\
    \frac{2}{1 + \sqrt{5}}  
    \end{bmatrix} = 0
    \]
    
    
    \item Using elimination
    \[[ \quad G \quad | \quad I \quad] \rightarrow [\quad I \quad | \quad G^{-1} \quad] \] we find the inverse of G:
    \[G^{-1} =  \begin{bmatrix}
-\frac{1}{2 \sqrt{5}} & \frac{1+\sqrt{5}}{4 \sqrt{5}}\\
\frac{1}{2 \sqrt{5}} & -\frac{1-\sqrt{5}}{4 \sqrt{5}} 
\end{bmatrix}\]
    
    
    
    \item We will check \(
    AG=G
    \begin{bmatrix}
    \lambda_{1} & 0 \\
    0 & \lambda_{2}
    \end{bmatrix}\) with the following calculations: 
    \[
    \begin{bmatrix}
    2 & 1  \\
    1 & 1  
    \end{bmatrix}
    \begin{bmatrix}
    1 & 1 \\ 
    \frac{2}{1 - \sqrt{5}} & \frac{2}{1 + \sqrt{5}}  
    \end{bmatrix} = 
    \begin{bmatrix}
    \frac{4- 2\sqrt{5}}{1 - \sqrt{5}} & \frac{4+ 2\sqrt{5}}{1 + \sqrt{5}}   \\
    \frac{3- \sqrt{5}}{1 - \sqrt{5}} & \frac{3+ \sqrt{5}}{1 + \sqrt{5}}  
    \end{bmatrix} = \begin{bmatrix}
    \frac{3- \sqrt{5}}{2} & \frac{3+ \sqrt{5}}{2}   \\
    \frac{3- \sqrt{5}}{1 - \sqrt{5}} & \frac{3+ \sqrt{5}}{1 + \sqrt{5}}  
    \end{bmatrix}\]
    \[
    \begin{bmatrix}
    1 & 1 \\ 
    \frac{2}{1 - \sqrt{5}} & \frac{2}{1 + \sqrt{5}}  
    \end{bmatrix} \begin{bmatrix}
    \frac{3- \sqrt{5}}{2} & 0  \\
    0 & \frac{3+ \sqrt{5}}{2}  
    \end{bmatrix} = 
    \begin{bmatrix}
    \frac{3- \sqrt{5}}{2} & \frac{3+ \sqrt{5}}{2}   \\
    \frac{3- \sqrt{5}}{1 - \sqrt{5}} & \frac{3+ \sqrt{5}}{1 + \sqrt{5}}  
    \end{bmatrix}\]
    
    
    \item Because G is invertible, we have:
    \[ (6):
    AG=G
    \begin{bmatrix}
    \lambda_{1} & 0 \\
    0 & \lambda_{2}
    \end{bmatrix}  \iff
    A=G
    \begin{bmatrix}
    \lambda_{1} & 0 \\
    0 & \lambda_{2}
    \end{bmatrix}
    G^{-1} \]
    
    \item We will prove this using induction:
    \begin{description}
    \item[Induction Hypothesis:] \(P(k)\): For any \(k \in \mathbb{N}\), \(A^k = G
    \begin{bmatrix}
    \lambda_{1}^{k} & 0 \\
    0 & \lambda_{2}^{k} \end{bmatrix} G^{-1}\). 
    \item[Base Case:]
      We have already shown \(P(1)\) is true.
    \item[Induction Step:] Assume \(P(k-1)\) is true, we will show \(P(k)\) is also true. 
      \[A^k = A^{k-1}A = G 
    \begin{bmatrix}
    \lambda_{1}^{k-1} & 0 \\
    0 & \lambda_{2}^{k-1}
    \end{bmatrix} G^{-1} G 
    \begin{bmatrix}
    \lambda_{1} & 0 \\
    0 & \lambda_{2} \end{bmatrix} G^{-1}= G 
    \begin{bmatrix}
    \lambda_{1}^{k} & 0 \\
    0 & \lambda_{2}^{k} \end{bmatrix} G^{-1}  \]
  \end{description}
    
    
     
    \item We have: 
    \[A^2 = 
    \begin{bmatrix}
    5 & 3 \\
    3 & 2
    \end{bmatrix}, \quad A^3 = 
    \begin{bmatrix}
   13 & 8 \\
    8 & 5
    \end{bmatrix}\] and 
    \[G
    \begin{bmatrix}
    \lambda_{1} & 0 \\
    0 & \lambda_{2}
    \end{bmatrix}
    G^{-1} = 
    \begin{bmatrix}
    1 & 1 \\ 
    \frac{2}{1 - \sqrt{5}} & \frac{2}{1 + \sqrt{5}}  
    \end{bmatrix} \begin{bmatrix}
    \frac{3- \sqrt{5}}{2} & 0  \\
    0 & \frac{3+ \sqrt{5}}{2}  
    \end{bmatrix}
    \begin{bmatrix}
    -\frac{1}{2 \sqrt{5}} & \frac{1+\sqrt{5}}{4 \sqrt{5}}\\
    \frac{1}{2 \sqrt{5}} & -\frac{1-\sqrt{5}}{4 \sqrt{5}} 
    \end{bmatrix}\]
    \[ G \begin{bmatrix}
    \lambda_{1}^2 & 0 \\
    0 & \lambda_{2}^2
    \end{bmatrix}
    G^{-1} = 
    \begin{bmatrix}
    5 & 3 \\
    3 & 2
    \end{bmatrix},
    \quad  
    G \begin{bmatrix}
    \lambda_{1}^3 & 0 \\
    0 & \lambda_{2}^3
    \end{bmatrix}
    G^{-1} =
    \begin{bmatrix}
   13 & 8 \\
    8 & 5 
    \end{bmatrix}
     \]
    
    
    
    
    
    
    
\end{enumerate}


\end{document}