\documentclass{article} %
\title{Chapter 7}
\author{Taro Kyodai}
\usepackage{} %

% if you type a letter with  the shift key located  in the right hand side of your japanese keyboad, the keyboard setting will change in VDI
% if the keyboad setting has chaned, pleae recover the keyboad setting with the button at the right upper corner of this VDI showing such as ja, en and so on.
% this is a system probelem. thank you for your understanding.


\begin{document}

\maketitle

\subsection*{7.2 Mathematical mode}

% mathematical expression should enclosed with $

%1st example
Let $f$ be the function defined by $f(x) = 3x + 7$, and
let $a$ be a positive real number.

%2nd example
Let $f$ be the function defined by
$
f(x) = 3x + 7
$
, and let $a$ be a positive real number.

%3rd example
Let
$
f
$
be the function defined by
$
f(x) = 3x + 7
$
, and let
$
a
$
be a positive real number.

% you can use \begin{eqation} and \end{equation} to number equation.

\begin{equation}
  f(x) + g(x) = 4x +11
\end{equation}

\begin{equation}
  f(x)g(x) = 3x^2 + 19x + 28
\end{equation}

% you can use \[ \] to not number equation.

\[
  f(x) + g(x) = 4x +11
\]

\[
  f(x)g(x) = 3x^2 + 19x + 28
\]

% letters are set in italic
% ' has a special extra meaning

\subsection*{7.3 Characters in mathematical mode}

$u' + v''$

% the extra spaces will not affect on the final result of spacing

$u         v + w =x$

$uv+w=x$

$uv + w = x$

% when you write equations, it is better to give proper space to distinguish each letters 

% the characters in mathematics mode should be typed with \

\[
\# \$ \% \& \_ \{ \} \backslash |
\]

\subsection*{7.4 Superscripts and subscripts}

\[
ds^2 = dx_1^2 + dx_2^2 + dx_3^2 - c^2 dt^2  
\]


\[
ds^2 = dx^2_1 + dx^2_2 + dx^2_3 - c^2 dt^2  
\]

$x^17 -1$

$x^{17}-1$

%you will find out the necessity of curly brackets to use intended expression

%$s^n^j$ %the folowing message is the error you will get with the left equation
% latex can not tell which to apply in superscript: ambiguous
%! Double superscript.$
%l.69 $s^n^ %
%          j$
%? %
% the following equations are included as a possible output
          
$s^{n j}$
$s^{n^j}$

\subsection*{7.5 Greek letters}

% Greek letters are easy to type in math mode

$\alpha,\beta,\gamma, \Gamma$
%while stating mathematical mode,  just type the spell with backslash ahead

$A = \pi r^2$

% it is better to give a space between greel letter and the following equation to distingishu them

\subsection*{7.6 Mathematical symbols}

$a_1+a_2+a_3+\cdots$ % \cdots is necessary to express series expression 
$a_1+a_2+a_3+\ldots$ % wrong use of \ldots


\subsection*{7.7 Standard fucnctions}

\[
\cos(\theta+\phi)=\cos \theta \cos \phi - \sin \theta \sin \phi
\]

\[
cos(\theta+\phi)=cos \theta cos \phi - sin \theta sin \phi
\]

% without backslash, the letter gets italic
% in correct expression of cos, sin, tan, we will use non-italic ones


\subsection*{7.8 Fractions and roots}

$\frac{numerator}{denominator}$

% you need to use it with mathematical mode, of course.

The function $f$ is given by
\[
f(x) = 2x + \frac{x-7}{x^2 + 4}
\]
for all real numbers $x$.

% to express square roots, \sqrt{expression}

The roots of a quadratic polynomial $a x^2 + bx + c$ with
$a \neq 0$ are given by the formula
\[
\frac{-b \pm \sqrt{b^2 - 4ac}}{2a}
\]

% to express an nth root, \sqrt[n]{expression}

The roots of a cubic polynomial of the form $x^3 - 3px -2q$
are given by the formula
\[
\sqrt[3]{q + \sqrt{q^2 - p^3}} + \sqrt[3]{q - \sqrt{q^2 - p^3}}
\]
where the values of the two cube roots must are chosen
so as to ensure that their product is equal to $p$.

\subsection*{7.9 Ellipsis}

Elipsis is produced mathematic mode using the control sequences with ldots and cdots. Carefully check the appeared potisions.

% use dots expression properly with the following example,

\[
f(x_1, x_2, \ldots, x_n) = x_1^2 + x_2^2 + \cdots + x_n^2 
\]

% cdots for omitting series

\subsection*{7.10 Brackets and norms}

The frequently used brackets are shown.\\

Let $X$ be a Banach space and let $f \colon B \ to \textbf{R}$
be a bounded linear functional on $x$. The \textit{norm} of
$f$, denoted by $\|f\|$, is defined by
\[
\|f\| = \inf \{K \in [0, +\infty) :
|f(x)| \leq K \|x\| \mbox{ for all } x \in X \}
\]

% for sinlge |, just type | without backslash. See 7.6
% pay attention to each bracket relationship

% for larger delimiters, we need to state special parenthesis with
% \left( expression \right): this function automatically search
% sufficient size of parehthesis

\[
f(x,y,z) = 3y^2 z \left( 3 + \frac{7x+5}{1+y^2} \right)
\]



% nesting pairs of delimiters

\[
\left| 4 x^3 + \left( x + \frac{42}{1+x^4} \right) \right|
\]

% with \left. and \right., we can obtain null delimiters
% null means ``nothing to output'' in programming

\[
{\left. \frac{du}{dx} \right| }_{x=0}
\]

\subsection*{7.11 Matrices and other arrays in LaTeX}

The \emph{charcteric polynomial} $\chi(\lambda)$ of the
$3 \times 3$~matrix % ~ to give a space
\[
\left( \begin{array}{ccc}
  a & b & c \\
  d & e & f \\
  g & h & i \end{array} \right)
\]
is given by the formula
\[
\chi(\lambda) = \left| \begin{array}{ccc}
  \lambda - a & -b & -c \\
  -d & \lambda - e & -f \\
  -g & -h & \lambda - i \end{array} \right|
\]

% 1: to make larger parethesis, \left( ... \right) is used
% 2: input the contents of matrix with & and \\
% 3: {ccc} means the typesetting in matrix
% for example, you can control typesetting like this;

\[
\begin{array}{lcr}
    \mbox{First number} & x & 8 \\
    \mbox{Second number} & y & 15 \\
    \mbox{Sum} & x + y & 23 \\
    \mbox{Difference} & x - y & -7 \\
    \mbox{Product} & xy & 120 \end{array}
\]

% just note: without \mbox{} you will get a italice letters in the table
% because the following table is availabel in mathematical mode

\[
\begin{array}{lcr}
    First number & x & 8 \\
    Second number & y & 15 \\
    Sum & x + y & 23 \\
    Difference & x - y & -7 \\
    Product & xy & 120 \end{array}
\]

\subsection*{7.12 Derivatives, limits, sums and integrals}

% for total differential, you can write with the learned things so far

\[
\frac{du}{dt}~and~\frac{d^2 u}{dx^2}
\]

% for partial differential, we will use \partial
% with the example of the Heat Equation

\[
\frac{\partial u}{\partial t}
= \frac{\partial^2 u}{\partial x^2}
+ \frac{\partial^2 u}{\partial y^2}
+ \frac{\partial^2 u}{\partial z^2}
\]

% to obtain mathematical expressions such as: lim, inf, sup

\[
\lim_{x \to + \infty}, \inf_{x>s} \mbox{and} \sup_K
\]

% example of combination;

\[
\lim_{x \to 0} \frac{3x^2 + 7x^3}{x^2 + 5x^4}=3
\]

% example os summation
% to express sum, we will use superscript and subscript function

\[
\sum_{i=1}^{2n}
\]

% in the textbook, the backslash ahead of sum is lack, sorry for that.

\[
\sum_{k=1}^n k^2 = \frac{1}{2} n (n+1)
\]

% finally, integrals in mathematical mode

\[
\int_a^b f(x)\,dx
\]

\[
\int_a^b f(x)dx % to give a correct appearance, we will use \, beofre the d
\]


% to express integrals, we will use superscript and subscript function, too

\[
\int_0^{+\infty} x^n e^{-x} \,dx = n!
\]
\[
\int \cos \theta \,d\theta = \sin \theta
\]
\[
\int_{x^2 + y^2 \leq R^2} f(x,y) \,dx\,dy
= \int_{\theta=0}^{2\pi} \int_{r=0}^R
f(r\cos\theta,r\sin\theta) r\,dr\,d\theta
\]

%Otsukare-sama deshita!

% for today's homework, you can copay & paste the texts in the tex file.
% We will check your typesetting skill and understanding of the mathematical mode

\end{document}
