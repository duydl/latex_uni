\documentclass[11pt]{exam}
\usepackage{amsmath, amsthm}  
\usepackage{enumitem}
\usepackage[T1]{fontenc}
\usepackage[bitstream-charter]{mathdesign}
\usepackage{parskip}
\usepackage{pgfplots}
\usepackage{booktabs}
\usepackage{tabularx}
\pgfplotsset{soldot/.style={color=blue,only marks,mark=*}} \pgfplotsset{holdot/.style={color=blue,fill=white,only marks,mark=*}}
\pgfplotsset{compat=1.17} 
% Theorems
\theoremstyle{definition}
\newtheorem*{defi}{Definition}
\newtheorem*{thm}{Theorem}

% Special sets
\newcommand{\C}{\mathbb{C}}
\newcommand{\CP}{\mathbb{CP}}
\newcommand{\GG}{\mathbb{G}}
\newcommand{\N}{\mathbb{N}}
\newcommand{\Q}{\mathbb{Q}}
\newcommand{\R}{\mathbb{R}}
\newcommand{\RP}{\mathbb{RP}}
\newcommand{\T}{\mathbb{T}}
\newcommand{\Z}{\mathbb{Z}}
\renewcommand{\H}{\mathbb{H}}

\pagenumbering{gobble}
\newcounter{questionCounter}
\newenvironment{numedquestion}[0]{%
	\stepcounter{questionCounter}%
    \vspace{.2in}%
        \ifx\writtensection\undefined
        \noindent{\bf \questiontype \; \arabic{questionCounter}. }%
        \else
          \if\writtensection
          \noindent{\bf \questiontype \; \arabic{questionCounter}. }%
          \else
          \noindent{\bf \questiontype \; \writtensection.\arabic{questionCounter} }%
        \fi
    \vspace{0.3em} \hrule \vspace{.1in}%
}{}



\footer{}{\thepage}{}
\renewcommand{\frac}{\dfrac}

\usepackage{multicol}
\newcommand{\myclass}{Calculus with Exercises A}
\newcommand{\myname}{DO LE DUY}
\newcommand{\myhwtype}{Report 07}
\newcommand{\questiontype}{Problem}
\newcommand{\writtensection}{7}
\headrule
\header{{\myname}}%
{\emph{\myclass}}%
{\textbf{\myhwtype }}
\DeclareMathOperator{\arcsec}{arcsec}
\begin{document}
\thispagestyle{empty}
\begin{center}
  {\Large \textbf{\myclass{} — \myhwtype{} }} \\
  {\myname{} } \\
  \today
\end{center}

\begin{numedquestion}
  
\begin{enumerate}[label={(\arabic*)}]
    \setcounter{enumi}{0}
  \item 
  \begin{align*}
  f'(x) &= (xe^{-x})\cos(xe^{-x}) \\
  &= (e^{-x} - xe^{-x})\cos(xe^{-x})
  \end{align*}
  \item 
  \begin{align*}
  f'(x) &= 4(x^2 + \cos^3 x)'(x^2 + \cos^3 x)^3 \\
  &= 4(2x - \sin x \cos^2 x)(x^2 + \cos^3 x)^3
  \end{align*}
  \item 
  \begin{align*}
  f'(x) &= \frac{-2x \sin(x^2)\sin^2(x)}{(1 + x^2)^2} + \frac{2x \cos(x^2)\sin^2(x)}{(1 + x^2)} + \frac{ 2\sin(x^2)\sin(x)\cos x}{(1 + x^2)} \\
  &= 2 \sin x\left(-\frac{x \sin(x^2)\sin x }{(1 + x^2)^2} + \frac{x \cos(x^2)\sin x}{(1 + x^2)} + \frac{ \sin(x^2)\cos x}{(1 + x^2)}\right)
  \end{align*}
\end{enumerate}
\end{numedquestion}

\begin{numedquestion}
    \begin{enumerate}[label = {(\arabic*)}]
    % setcounter{enumi}{0}
    \item  
    \begin{align*}
      f'(x) &= \lim_{x \rightarrow 0} \frac{x\sqrt{|x|} - 0\sqrt{|0|}}{x - 0} \\
      &= \lim_{x \rightarrow 0}\sqrt{|x|} = 0
    \end{align*}
    \item \begin{align*}
      f'(x) &= \lim_ {x \rightarrow 0} \frac{x\cos \frac{1}{x} - 0}{x - 0} \\
      &= \lim_ {x \rightarrow 0} \cos \frac{1}{x}
    \end{align*}
    which does not exist. This could be proved using the two sequences $a_{n}=\frac{1}{2 n \pi} \text { and } b_{n}=\frac{1}{(2 n+1) \pi}$.
    \item \begin{align*}
      f'(x) &= \lim_ {x^3 \rightarrow 0} \frac{x\cos \frac{1}{x} - 0}{x - 0} \\
      &= \lim_ {x \rightarrow 0} x^2 \cos \frac{1}{x} = 0 \text { using Squeeze Theorem}
    \end{align*}
   
  \end{enumerate}
\end{numedquestion}
  
\begin{numedquestion}
  The $\sec x$ function has no inverse. To determine the domain of $\arcsec (x)$, we first have to select a restricted domain for $\sec x$, that is creating a new function $\sec x$ with domain  $[0, \pi]$. Then the domain of $\arcsec(x)$ is $(-\infty, -1] \cup [1, \infty)$. 
  The graph for $\arcsec(x)$:\\
\begin{tikzpicture}
  \begin{axis}[axis x line=middle, axis y line=middle, samples = 500,
    xmin=-6 - 1/2, xmax=6 + 1/2, xlabel=$x$,
    ymin=-100, ymax=210, ytick={-90, 90, 180}, ylabel=$y$,
    x post scale= 2
    ]
    \addplot[domain = -5:-1, color = red] {acos(1/x)};
    \addplot[domain = 1:5, color = red]   {acos(1/x)};
    \draw[dotted] (axis cs:-6,90) -- (axis cs:6,90);
  \end{axis}
\end{tikzpicture}

    \begin{align*}
      y = \arcsec x &\implies  \sec y = x \\
      & \implies \frac{d x}{d t} = \sec y \tan y \\
      & \implies \frac{d y}{d x} = \frac{1}{\sec y \tan y} = \frac{1}{x \tan y}\text{ as } \tan y = \pm \sqrt{1 - x^2} \text{ we will square the equation. }\\
      & \implies \left(\frac{d y}{d x} \right)^2 = \left(\frac{1}{(\sec y \tan y)^2}\right) = \left(\frac{1}{x^2 (x^2 - 1)}\right)\\
      & \implies \left|\frac{d y}{d x} \right| = \left(\frac{1}{|x| \sqrt{x^2 - 1}}\right)\\
    \end{align*}
    Since $\frac{d y}{d x} = \frac{1}{\sec y \tan y} = \frac{\cos^2 y}{\sin y}$, $\frac{d y}{d x}$ will have the same sign as $\sin y$, which is always positive. We conclude that:
\[\frac{d(\arcsec x)}{d x} = \frac{1}{|x| \sqrt{x^2 - 1}} = \left\{\begin{array}{l}
  \frac{1}{x \sqrt{x^2 - 1}}  \text { for } x \geq 1 \\
  \frac{-1}{x \sqrt{x^2 - 1}} \text { for } x \leq -1 \end{array}\right.\]
\end{numedquestion}

\begin{numedquestion}
  We have $p^{(k)}(x) = a_k k! + a_{k+1} \frac{(k+1)!}{1!} x + \dots + a_n \frac{n!}{(n-k)!}x^{n-k}$ \newline
  Therefore $p^{(k)}(0) = a_k k!$.
\end{numedquestion}


% \begin{numedquestion}
%   CASE $1: \alpha>1$
%   Let $f(x)=x^{\alpha}$ for $x \in[0, \infty)$ and $\alpha>1 .$ We wish to determine whether $f$ is uniformly continuous on $[0, \infty)$
%   From the mean value theorem, there exists a number $x^{*} \in\left(x, y\right)$ such that
%   \[
%   \left|f\left(y\right)-f\left(x\right)\right|=\left|f^{\prime}\left(x^{*}\right)\right|\left|y-x\right|=\alpha\left(x^{*}\right)^{\alpha-1}\left|y-x\right|
%   \]
%   For $\epsilon=\alpha / 2,$ choose any $\delta>0,$ however small. Then, for $x=1 / \delta^{1 /(\alpha-1)}$ and $y=\frac{1}{2} \delta+1 / \delta^{1 /(\alpha-1)},$ we see that
%   \[
%   \left|f\left(y\right)-f\left(x\right)\right|=\left|f^{\prime}\left(x^{*}\right)\right|\left|y-x\right|=\alpha\left(x^{*}\right)^{\alpha-1} \delta / 2 \geq \alpha / 2=\epsilon
%   \]
%   Therefore, $f(x)=x^{\alpha}$ fails to be uniformly continuous on $[0, \infty)$ when $\alpha>1$

%   \[
% \text { CASE } 2: \alpha<1 \text { and } x>x_{0}>0
% \]
% Then given any $\epsilon>0$
% \[
% \begin{array}{r}
% \left|f\left(y\right)-f\left(x\right)\right|=\alpha\left(x^{*}\right)^{\alpha-1}\left|y-x\right| \leq \alpha\left(x_{0}\right)^{\alpha-1}\left|y-x\right|<\epsilon \\
% \text { whenever }\left|y-x\right|<\delta=\frac{x_{0}^{1-\alpha}}{\alpha} \epsilon
% \end{array}
% \]
% Therefore, $f$ is uniformly continuous for $x \in\left[x_{0}, \infty\right)$ for all $x_{0}>0$ and $\alpha<1$
  
% \end{numedquestion}

\end{document}


% https://math.stackexchange.com/questions/2052082/fx-x-alpha-uniformly-continuous-on-0-infty-using-mvt
% https://www.youtube.com/watch?v=U-UmtZkCHwU&t=9s
% https://math.stackexchange.com/questions/569928/sqrt-x-is-uniformly-continuous
% Let $\epsilon>0 .$ Pick $\delta=\epsilon^{2} .$ Then for $|x-y|<\delta$ we have
% \[
% |\sqrt{x}-\sqrt{y}|^{2} \leq|\sqrt{x}-\sqrt{y}||\sqrt{x}+\sqrt{y}|=|x-y|<\epsilon^{2} \Longrightarrow|\sqrt{x}-\sqrt{y}|<\epsilon
% \]

% https://math.stackexchange.com/questions/135234/showing-fx-x4-is-not-uniformly-continuous?rq=1
% To show that it is not uniformly continuous on the whole line, there are two usual (and similar) ways to do it:
% 1. Show that for every $\delta>0$ there exist $x$ and $y$ such that $|x-y|<\delta$ and $|f(x)-f(y)|$ is greater than some positive constant (usually this is even arbitrarily large).
% 2. Fix the $\varepsilon$ and show that for $|f(x)-f(y)|<\varepsilon$ we need $\delta=0$
% First way:
% Fix $\delta>0,$ set $y=x+\delta$ and check
% \[
% \lim _{x \rightarrow \infty}\left|x^{4}-(x+\delta)^{4}\right|=\lim _{x \rightarrow \infty} 4 x^{3} \delta+o\left(x^{3}\right)=+\infty
% \]

% on the other hand this means we increase the complexity of our brain by unterstanding it